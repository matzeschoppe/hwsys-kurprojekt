% deckblatt.tex, 2019/03/15 
\documentclass[a4paper,12pt]{article}  
%Inhaltsverzeichnis in Kapitel und Sektionen aufgeteilt  \chapter{title} > \section{title} > subsection{title} \subsubsection{title}

% Pakete und Paket-Configs
\usepackage{times} % Times Roman als Standardschrift
\usepackage[ngerman]{babel} % neue deutsche Rechtschreibung und Trennung
\usepackage{fancyhdr} % spezielle Kopfzeilen
\usepackage[latin1]{inputenc} % Umlaute ��� auch normal benutzen und nicht maskieren
\usepackage[babel, german=quotes]{csquotes}
\usepackage{subfigure} % Figures divided into subfigures.
\usepackage{ifthen} % Ermöglicht ifthenelse und whiledo
\usepackage{amsfonts} % Extra mathematical symbols
\usepackage[rflt]{floatflt} % verbessertes floatfig, als um figure's fliesende texte
\usepackage[T1]{fontenc} % ?? aber notwendig für korrekte PDF-Metadaten
%\usepackage{longtable} % Support for tables longer than a page.
%\usepackage{a4wide} % Increases width of printed area of an a4 page.
%\usepackage{alltt} % verbatim environment except that \ and braces have their usual meanings.
\usepackage{listings} % Typeset source code listings using LaTeX.
\usepackage{moreverb} % bessere verbatim-umgebungen
\usepackage{graphicx}
\usepackage{pdfpages}
\usepackage{xcolor} % Farben Definierbar \definecolor{fhorange}{RGB}{255,153,0}
\usepackage{biblatex} % F�r das Erstellen eines Literaturverzeichnisses
\addbibresource{literatur.bib}  % f�r das erstellen des Literaturverzeichnisses empfielt sich die Software Mendeley, diese erm�glicht einfaches hinzuf�gen B�chern und das exportieren in eine .bib Datei 



\pagestyle{headings} % Kopf- und Fusszeilen
\pagenumbering{Roman} % Nummerierung der Seiten
\newcommand{\maximagewidth}{15cm} % maximal m�gliche Bildbreite

\setlength{\parindent}{0cm} % Einr�ckung am Abstzanfang
\setlength{\parskip}{5pt plus 2pt minus 1pt} % Abstand der Abs�tze zueinander
\frenchspacing % Kein Zusatzzwischenraum nach Satzzeichen

\setcounter{secnumdepth}{3} % Z�hlung bis paragraph 1.1.1
\setcounter{tocdepth}{3} % Inhaltsverzeichnis bis paragraph 1.1.1

%\makeglossary % Schreibe ein Glossar-File

\title{Titel der Arbeit}
\author{Name des Authors}
\date{\today}
\setcounter{section}{0}
\setcounter{secnumdepth}{3}
\definecolor{fhorange}{RGB}{255,153,0}
\begin{document}

\begin{titlepage}
	\thispagestyle{empty}
	\begin{center}
		\begin{minipage}{15cm}
			\begin{flushright}
				\includegraphics[width=4cm]{fhlogo.png} \hspace{1.5cm} \\
				\textbf{\large Hochschule \hspace{3.5cm} ~ \\ Augsburg } \large University of \hspace{1cm} ~ \\ Applied Sciences \hspace{0.2cm} ~
				\\~
			\end{flushright}
			
			\begin{flushleft}
				\textbf{\large \textcolor{fhorange}{Ausarbeitung}} \hspace{7.7cm} \textcolor{fhorange}{\large Fakult�t f�r}
				\vspace{0.1cm}
				\\~  \hspace{10.75cm} \textcolor{fhorange}{\large Informatik} 
			\end{flushleft}
			
			%\vspace{1cm}
			\begin{flushleft}
				{ \large Studienrichtung \\ 
					%	\vspace{0.2cm}
					M.Sc. Informatik
				}
				\vspace{0.1cm}
			\end{flushleft}
			\begin{flushleft}
				{
					\textcolor{fhorange}{\large \textbf{Mathias Schoppe \\
							Entwicklun eines integrierten Schaltkreises - Konzeptionierung und Umsetzung}}
				}
				\vspace{0.2cm}
			\end{flushleft}
			\begin{flushleft}
				{ \large Abgabe bei: Prof. Dr.-Ing. Gundolf Kiefer \\ 
					\vspace{0.1cm}
					Abgabe der Arbeit am: 09.07.2023 			
				}
			\end{flushleft}
			\begin{flushleft}
				%In Kooperation mit Firma: \\ Hsa-Digit \\ Betreuer: Hans Mustermann \\ 
				%\vspace{0.5cm}
				%\includegraphics[width=3cm]{hsa_digit.png} \hspace{5cm} 
				\tiny{\textcolor{gray}{
						\hspace{11.3cm} Hochschule f�r angewandte \\
						\hspace{11.3cm} Wissenschaften Augsburg \\
						\hspace{11.3cm} University of Applied Sciences \\ ~ \\
						\hspace{11.3cm} An der Hochschule 1 \\
						\hspace{11.3cm} D-86161 Augsburg \\ ~ \\
						\hspace{11.3cm} Telefon +49 821 55 86-0 \\
						\hspace{11.3cm} Fax +49 821 55 86-3222 \\
						\hspace{11.3cm} www.hs-augsburg.de \\
						\hspace{11.3cm} info@hs-augsburg.de \\} ~ \\
					\hspace{11.3cm} Fakult�t f�r Informatik \\
					\hspace{11.3cm} Telefon +49 821 5586-3450 \\
					\hspace{11.3cm} Fax ~~~ +49 821 5586-3499 \\ ~ \\
					\hspace{11.3cm} Verfasser der Ausarbeitung: \\
					\hspace{11.3cm} Mathias Schoppe \\
					\hspace{11.3cm} mathias.schoppe@hs-augsburg.de \\
				}
			\end{flushleft}
		\end{minipage}
	\end{center}
\end{titlepage}


% Zusammenfassung
\thispagestyle{empty}
\section*{Abstract}
\newpage
\tableofcontents  
% Wenn das Inhaltsverzeichnis aktualisiert werden muss 2 mal kompilieren, erst dann wird die Aktualisierung angezeigt
\newpage
\pagenumbering{arabic}
\section{Einleitung}
\subsection{Motivation}
\subsection{Zielsetzung}

\section{Grundlagen}
\subsection{Linear r�ckgekoppeltes Schieberegister}
\subsection{Hardware Timer}

\section{Konzept}
\subsection{Schaltkreis Entwurf}
\subsubsection{Systemebene}
%Generelles Konzept und Programmablauf (Ablaufdiagramm)
\subsubsection{Register-Transfer-Ebene}
%Algorithmus (Beschreibungssprache)
\subsubsection{Gatter-Ebene}
%Strukturdiagramm


\section{Umsetzung}
\subsection{Implementierung}
%Wokwi/Vhdl programm vorstellen
\

\section{Bewertung}
\subsection{Zieleinhaltung}
\subsection{Interpretation der Ergebnisse}

\section{Zusammenfassung und Ausblick}




\end{document}
